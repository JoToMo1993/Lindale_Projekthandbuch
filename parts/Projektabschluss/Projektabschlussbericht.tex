\section{Projektabschlussbericht}
\noindent
%vspace between the lines
\tabulinesep=1.3mm
\begin{longtabu} to \linewidth {|X[c]|X[c]|X[c]|X[c]|X[c]|}

\hline 
\rowfont{\large}
\multicolumn{5}{|c|}{ Projektabschlussbericht } \\ 
\hline 
sehr gut & gut \checkmark & okay & schlecht & sehr schlecht \\ 
\hline
  
\multicolumn{5}{|l|}{ \bfseries Gesamteindruck } \\ 
\hline    

\multicolumn{5}{|l|}{
\parbox{15cm}{ 
	\begin{itemize}
	   \item Projektziele nicht alle erreicht
	   \begin{itemize}
	       \item Groplanung und Modellierung abgeschlossen
	       \item Technologien ausgewählt
	   	   \item Entwicklung des Hauptprogramms erreicht
	   \end{itemize}
	   \item Noch nicht erreichte Ziele, wegen Verschiebung einer Klausur
	   \begin{itemize}
	     \item Entwicklung von Plugins
	     \item Testen der gesamten Applikation
	   \end{itemize}
	\end{itemize}
}
} \\
\hline 

\multicolumn{5}{|l|}{ \bfseries Reflexion: Erfüllung der geplanten Leistungen, Einhaltung der geplanten Termine} \\ 
\hline

\multicolumn{5}{|l|}{ 
\parbox{15cm}{
\begin{itemize}
  \item Erreichte Meilensteine
  \begin{itemize}
    \item Grobplanung und Modellierung abgeschlossen
    \item Technologien ausgewählt
    \item Entwicklung des Hauptprogramms mit Verspätung erreicht
  \end{itemize}
  \item Erreichte Leistungen
  \begin{itemize}
    \item viel Wissen durch ausgiebige Recherche erarbeitet
    \item Gute Planung und modellierung der Applikation
    \item Teilweise Implementierung des Projektes
  \end{itemize}
\end{itemize}
}
} \\
\hline

\multicolumn{5}{|l|}{ \bfseries Reflexion: Projektumweltbeziehungen, Beziehungen zu anderen Projekten } \\ 
\hline 
\multicolumn{5}{|l|}{
\parbox{15cm}{
\begin{itemize}
  \item Projektumweltbeziehung
  \begin{itemize}
    \item Die Anzahl der Dozenten hat sich im Verlauf des Projektes von zwei Ansprechpartner auf vier Ansprechpartner erhöht
  \end{itemize}
  \item Beziehung zu anderen Projekten
  \begin{itemize}
    \item Die Entwicklung von Java 8 durch das OpenJDK Projekt verläuft Planmäßig
  \end{itemize}
\end{itemize}
}
} \\
\hline 

\multicolumn{5}{|l|}{ \bfseries Reflexion: Teamarbeit, Einsatz von Projektmanagement } \\ 
\hline 
\multicolumn{5}{|l|}{
\parbox{15cm}{
\begin{itemize}
  \item hoher Einsatz von Projektmanagement Methoden durch Lehrveranstaltung zum Projektmanagement
  \item Projekt unter permanenter Kontrolle aller Teammitglieder
  \item Gute Teamarbeit und Koordinierung durch
  \begin{itemize}
    \item Projektteamleiter
    \item Einsatz von Kommunikationstechnologie
    \item klare Terminregelungen
  \end{itemize}
  \item hoher Wissensaustausch über eigenes Wiki
  \item ausgegliechene Aufgabenverteilung auf alle Mitglieder
  \item insgesamt sehr gute Teamarbeit
\end{itemize}
}
} \\
\hline  

\multicolumn{5}{|l|}{ \bfseries Zusammenfassende Erfahrungen für andere Projekte } \\ 
\hline 

\multicolumn{5}{|l|}{
\parbox{15cm}{
\begin{itemize}
  \item lernen von Projektmanagement Methoden anhand des Projektes kann in Zukunft genutzt werden
  \item Erfahrung mit sehr neuen, teilweise noch im Beta-Stadium befindlicher, Technologien
  \item Datenbank modellierung und programmierung wurde durch LV's erarbeitet
  \item kennen lernen von Git als Code-Versionierungs-Verwaltung für Teams
  \item erste Erfahrung mit Software Engeeniering von größeren Anwendungen
\end{itemize}
}
} \\
\hline 

\multicolumn{1}{|l}{Version 1.0} & \multicolumn{2}{l}{Datum 06 Jan, 2014} & \multicolumn{2}{l|}{Ersteller: Lindale-Team } \\
\hline

%\subsection{Seite 391 Happy Projects!, Roland Gareis, MANZ 2006 }
\end{longtabu}
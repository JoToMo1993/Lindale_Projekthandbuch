\section{Projektfortschrittsbericht}
\noindent
%vspace between the lines
\tabulinesep=1.3mm
\begin{longtabu} to 14cm {|X[c]|X[c]|X[c]|X[c]|X[c]|}
\hline 
\rowfont{\large}
\multicolumn{5}{|c|}{ Lindale Projektfortschrittsbericht} \\ 
\hline 
sehr gut & gut \checkmark & okay & schlecht & sehr schlecht \\ 
\hline 
\multicolumn{5}{|l|}{ \bfseries Gestamtstatus des Projektes } \\ 
\hline 
\multicolumn{5}{|l|}{
\parbox{15cm}{ 
	\begin{itemize}
	   \item Projekt liegt im Zeitrahmen
	   \item Hinzugekommenes Know-how durch Datenbankprogrammierungs Lehrveranstaltung und Seminar
	   \item Angespannte Arbeitssituation aller Teammitglieder
	   \item Verteilung von Inhalten des Projekthandbuches → Informations und Datenmisstand innerhalb des Projektes
	\end{itemize}
}
} \\
\hline 
\multicolumn{5}{|l|}{ \bfseries Status: Projektziele und Betrachtungsobjekte } \\ 
\hline 
\multicolumn{5}{|l|}{
\parbox{12cm}{
	\begin{itemize}
	   \item Zieladaptierung: weglassen von Features wie:
	   \begin{itemize}
	     \item Logische und Physikalische Collection
	     \item Bewertungssystem
	   \end{itemize}
	   \item Adaptierung der Nebenziele war notwendig dennoch $\rightarrow$ Status: \gut
	\end{itemize}
}
} \\
\hline
\multicolumn{5}{|l|}{ \bfseries Status: Projektleistungsfortschritt } \\ 
\hline 
\multicolumn{5}{|l|}{
\parbox{12cm}{
    \begin{itemize}
      \item Arbeitspakete fortschritt:
     
	  \begin{description}
	     \item[1.2] Erfolgreich umgesetzt
	     \item[1.3] Recherche für die Hauptprogramm Entwicklung abgeschlossen
	     \item[1.4.3] Spezifizieren für die Entwicklung des Hauptprogramms abgeschlossen
	     \item[1.4.5] Standards soweit festgelegt, das entwickelt werden kann
	     \item[1.5.5] Datenbank Modellierung abgeschlossen
	  \end{description}
	  
	  \item Verzug mit dem modellieren der GUI des Hauptprogrammes $\rightarrow$ Status: \okay
	\end{itemize}
  }
} \\
\hline 
\multicolumn{5}{|l|}{\bfseries Status: Projekttermine } \\ 
\hline 
\multicolumn{5}{|l|}{
\parbox{12cm}{
	\begin{itemize}
	   \item Verzögerung bei der Entwicklung und Modellierung des Hauptprogramms durch neu hinzugewonnenes Wissen
	   \item Wegen Verzögerung beim modellieren $\rightarrow$ Status: \okay
	\end{itemize}
  }
} \\
\hline 
\multicolumn{5}{|l|}{ \bfseries Status: Projektumwelt, Beziehungen zu anderen Projekten } \\ 
\hline 
\multicolumn{5}{|l|}{
\parbox{12cm}{
	\begin{itemize}
	   \item Hinzugekommener Dozent als Experte zwischen Datenbank und Anwendungsprogramm
	   \item Starke Wissenserweiterung $\rightarrow$ Status: \sehrGut
	\end{itemize}
}
} \\
\hline 
\multicolumn{5}{|l|}{ \bfseries Maßnahmenkatalog } \\ 
\hline 
\multicolumn{5}{|l|}{
\parbox{12cm}{
	\begin{itemize}
	  \item Adaptieren der Nebenziele
	  \item Neu ausrichtung des Projektes mithilfe des gewonnenen Informationen
	  \item Zusammenführen der Projekthandbuch Informationen
	  \item Einpflegen von Informationen in das Informationssystem
	  \item Information aller Teammitglieder und Umsetzung läuft $\rightarrow$ Status: \sehrGut
	\end{itemize}
}
} \\
\hline 
\multicolumn{5}{|l|}{ \bfseries Status: Anhang } \\ 
\hline 
\multicolumn{5}{|l|}{
\parbox{12cm}{
    \begin{itemize}
	   \item Projekt Score Card
	\end{itemize}
}
} \\
\hline 
\multicolumn{1}{|l}{Version 1.1} & \multicolumn{2}{l}{Datum 6 Dez, 2013} & \multicolumn{2}{l|}{Ersteller: Lindale-Team } \\
\hline

\end{longtabu}
